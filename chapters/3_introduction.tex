% What do I want to say:
% Start with why graphs are useful 
% Then talk about graph databases 
The size of graphs being used in both academia and industry is increasing at a
rapid rate, and as a result, it has become common for graphs to have tens of
billions of edges\cite{sahu2017ubiquity}. Furthermore, this trend is expected to
continue as the amount of data collected and processed increases. A recent
survey shows that scalability is the primary concern of graph database
users\cite{sahu2017ubiquity}. As this scale
increases, decoupling storage from graph query evaluations using distributed
cloud storage services like AWS S3\cite{awsS3} may be
more cost-efficient. These systems provide durable data storage, high
throughput, and a pay-as-you-go model where you only pay for the data you store
and the number of requests you make. Making use of these characteristics would
enable graph databases to provide more flexible scaling and alleviate 
the costs related to redundancy. Therefore, it is worthwhile to explore
the feasibility of such database architectures.

\medskip
The idea of having separate storage and compute has been studied in the context
of relational databases. In 2008, Brantner et al\cite{brantner2008building}
described how a relational database could be built on top of AWS S3 while still
providing atomicity, isolation, and durability. Snowflake\cite{snowflake} was
one of the first databases which fully realized the idea of having
separate storage and compute layers. This idea was adopted by other databases
like AWS Aurora\cite{verbitski2017amazon} in 2017 and more recently by
Neon\cite{neonPostgres}. Despite the existence of so many relational databases
which have separate storage and compute layers, this idea remains unexplored for
native graph databases.

\medskip
Although there are many native graph databases that enable scaling out
using sharding\cite{besta2023demystifying}, the compute and storage are
still tied together. Unlike such architectures, this paper explores an
architecture where storage of raw graphs is delegated to a distributed cloud
store and, memory of compute instances is only used to process queries and store
intermediate results.

\medskip
There are two main challenges when considering databases with separate compute
and storage: Distribution transaction management and access latency. The first
challenge, although more intricate, has been studied in the context of
relational databases\cite{brantner2008building} and can be extended for other
systems that follow this general architecture. Therefore, in this paper, we
explore how to reduce the access latency for graph-specific operations in
the context of graphs stored on distributed file systems.

\medskip
To study the impact of our techniques on access latency, we use
Breadth-first search (BFS) and Depth-first search (DFS) as two algorithms to
benchmark our performance. These two traversal algorithms are the most
commonly used algorithms for performing graph traversals\cite{sahu2017ubiquity}.
These traversal algorithms will have a bounded depth and thus only
access a small part of the graph. On the spectrum of graph algorithms from OLTP
to OLAP as described by Besta et al.\cite{besta2023demystifying}, these 
traversal algorithms lie
more on the OLTP side as the portion of the graph explored by a single traversal
would be much smaller than the size of the graph.

\medskip
To minimize the latency, we will first propose a graph storage format that
would provide a gain granular access over the graph stored in AWS S3. Then, we
would extend some of the existing caching techniques to lower the latency of
graph access. Finally, we will describe the cases where it is viable and more
cost-efficient to use an architecture that separates storage from compute.

\section{Research Objectives}
In this paper, we provide an initial analysis of how distributed storage services can 
be used to query large graphs. While using these distributed storage services, the primary
issue that needs to be addressed is latency. The latency of accessing data in a networked
distributed system involves communication over the network which is at least three orders of
magnitude more than accessing data from local storage. In return for this increased latency,
we get virtually unlimited throughput as read operations are distributed across a cluster 
of thousands of physical machines if not more. Therefore, in this paper, we evaluate ways to
reduce the latency of accessing graphs and then quantify the performance
benefits of increased throughput capacity by employing parallelism. We will focus on the 
performance of two commonly used graph traversal algorithms: Breadth-first search 
(BFS) and Depth-first search (DFS). More formally, the first research
objective is as follows:
\begin{displayquote}
    \textbf{RO 1:} Gauge the effectiveness of caching and prefetching techniques
    on the latency of graph access for graph traversals and the effectiveness of
    using parallelism to increase throughput of these traversals.
\end{displayquote}

\medskip
Apart from reducing the latency, we will also discuss how the cost model of
distributed storage engines differs from the traditional model of coupled
compute and storage. For storage engines like S3, we are charged for every
gigabyte of storage and we are charged for every request that we make on that
storage. On the other hand, in the case of an SSD/HDD storage unit, you pay for a
certain amount of storage and there is no additional cost for accessing the
storage. Therefore, we aim to provide a model to help users choose between one
form of storage over the other. More formally, the second research objective is
as follows:
\begin{displayquote}
    \textbf{RO 2:} Provide a cost model to help users decide whether using S3
    instead of SSD/HDDs might be more cost effective.
\end{displayquote}

\section{Research contributions}
In line with the research objectives mentioned in the previous section, the 
unique contributions of this research are as follows:
\begin{enumerate}
    \item We provide the details of how we can perform traversals on large
        graphs using a distribution cloud storage service. The architecture for
        separating compute and storage was already proposed by Brantner et
        al.\cite{brantner2008building} in 2008, however, there has been no
        research on the viability of this architecture for graph databases. We
        answer this question in Chapter \ref{chapter:systemArchitecture} and
        discuss its performance in Chapter \ref{chapter:evaluation}.
    \item We provide a cost model to enable users, and developers of graph
        databases to check if using S3 or similar tools would fit their use case.
        The authors are not aware of any research which answers this question.
        We answer this question in Section \ref{sec:costModel}.
\end{enumerate}


\section{Structure of the thesis}

\medskip
Chapter \ref{chapter:preliminaries}, provides the necessary background for
this thesis. Section \ref{sec:distributedStorage} talks about the
history, architecture, and characteristics of distributed storage systems. We
also give a brief overview of the capabilities of AWS S3, which is the service
we use in this thesis. Then
Section \ref{sec:serverlessArch} discusses the advantages of database
architectures which separate compute and storage. Finally, 
Section \ref{sec:cachingDistSys} discusses the caching and prefetching
strategies that were employed in this thesis to lower the latency of graph
traversals.

\medskip
Chapter \ref{chapter:systemArchitecture}, introduces our architecture for
performing traversals. We start by providing a description of each component and
its responsibilities in Section \ref{sec:componentOverview}. 
Then Section \ref{sec:baseline} provides a baseline implementation which will be useful for comparing
the impact of the improvements made in the subsequent sections. Finally, 
Sections \ref{sec:graphAccess} and \ref{sec:parallelAlgorithms} 
describe the details of our proposed architecture
which enables low-latency traversals for large graphs.

\medskip
In Chapter \ref{chapter:evaluation}, we present benchmarks showcasing the
effectiveness of our implementation and the underlying system architecture. We begin this chapter by
comparing the performance with the baseline solution in
Section \ref{sec:cmpBaseline}. Then in Section \ref{sec:cmpOtherTools}, we
compare the performance of our solution with Neo4j and Apache Flink. This
section highlights various characteristics of different types of tools(GDBMS,
RDBMS, Big Data tools, and Custom Solutions) and the areas in which they 
are suitable. 

\medskip
Chapter \ref{chapter:discussion} provides concrete use case for the use of our
proposed architecture and considers some threats to the credibility of this
thesis. Section \ref{sec:costModel} provides a formal cost model for reasoning
about the cost difference between using AWS S3 versus coupled storage. In this
section, we also provide use cases where using one architecture over the other
may be more cost efficient. In Section \ref{sec:threats}, we discuss the threats
to the credibility of this research. This section talks about how some of the
aspects of a graph database that we have not considered in this thesis may
impact the viability of AWS S3 as a graph storage layer. 

\medskip
Finally, Chapter \ref{chapter:conclusions} concludes the thesis and
suggests possible directions for future work. This section contains information
about how we may be able to extend this work to reach a point where we have a
fully functioning graph database whose storage resides in S3.
