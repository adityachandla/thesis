As the size of graphs being queried by graph databases increases, coupling 
storage and compute increases the cost and limits the flexibility of
traditional graph database management systems (GDBMS). As part of this thesis,
we explore the viability of an architecture where the compute and storage are
independent. This independence is achieved using a distributed cloud storage
service (AWS S3) which provides bottomless storage and theoretically unlimited
throughput. Using this distributed cloud storage solution, we evaluate the
latency of running two common graph traversal algorithms i.e breadth first
search (BFS) and depth first search (DFS). We then compare this latency with
some other systems which may be used to perform BFS and DFS on large datasets. 
Finally, we provide concrete use cases where using such a decoupled system might
be better suited. 
